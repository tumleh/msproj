\documentclass{IEEEtran}%[conference]{IEEEtran}%\documentclass[10pt]{article}%{article}{report}{letter}{book}{proc}{slides}%
\usepackage{amsmath}
\usepackage{amssymb}
\usepackage{amsthm}
\usepackage{hyperref}
\usepackage{graphicx}
\usepackage{wasysym}
\usepackage{skull}

\hypersetup{colorlinks=true,linkcolor=blue,citecolor=blue}

%\textwidth=5.5in
%\oddsidemargin=0.5in
%\evensidemargin=0.0in
%\textheight=8.0in
%\topmargin=0.0in



\begin{document}

\title{Delay Comparison of Different Switch Architectures}



% author names and affiliations
% use a multiple column layout for up to three different
% affiliations
%\author{Stephan Adams and Libin Jiang}
%\IEEEauthorblockA{Department of Electrical Engineering and Computer Science\\
%University of California, Berkeley\\
%\{shadams, jnchang\}@eecs.berkeley.edu}
\author{\IEEEauthorblockN{S. H. Adams, L. Huang, A. Parekh, and J. Walrand \\}}
%\IEEEauthorblockA{Department of Electrical Engineering and Computer Science\\
%University of California, Berkeley\\
%\{shadams,jnchang\}@eecs.berkeley.edu}}


% conference papers do not typically use \thanks and this command
% is locked out in conference mode. If really needed, such as for
% the acknowledgment of grants, issue a \IEEEoverridecommandlockouts
% after \documentclass

% use for special paper notices
%\IEEEspecialpapernotice{(Invited Paper)}

% make the title area
\maketitle

\section{Next Step:}
\begin{itemize}
\item pick suitable parameters
\item Build Experiments?
\end{itemize}
\section{To Do}
\begin{itemize}
\item Build Simulator: Now -- Oct 15th?
\item Perform Experiments: Oct 15th -- November 15th?
\item Write Paper: Now -- December 15th?
\item Time Line
\end{itemize}

\section{Simulator Pieces:}
\subsection{Crossbar}
\subsection{Scheduling Modules}
\subsection{Input Interface}
\subsection{Packet Generation}
\subsubsection{TCP Layer}
\subsubsection{Application Layer}
\subsection{Experiment Functions}
\subsection{Analysis Tools}
\subsubsection{Logs}
\subsubsection{Data Collection}
\subsubsection{Unit Tests}
\begin{itemize}
\item state is properly initialized
\item event merging
\item only real schedules are generated
\item right averages are attained
\item TCP waveforms are right when the channel is predictable
\item Event merge.
\end{itemize}
\subsubsection{Matlab Data Display Tools}
\subsection{Setting Parameters}


\section{Perform Experiments}

Try to mirror the experiments with the previous simulator as much as possible.
\subsection{What kind of packet distributions?}
Try bimodal distribution with high probability around a big size and small size with slight variations around the means.  Also single size with some slight variations.  Finally try without any of the small variations.
\subsection{What kind of Flow patterns?}
Start with those we have decided model the datacenter.
\subsection{Throughput Vs Delay Graphs}
\subsubsection{What are good parameters for QCSMA?}
\subsubsection{Number of Slots for SLIP}
\subsection{TCP performance}
\subsection{Performance Variation as packet sizes become more predictable}

\section{Write Paper}
\subsection{Would like to show: QCSMA is better than SLIP by x\%}
\subsection{Would like to show: Equal Length vs Variable Length Packets makes a big difference!}
\subsection{Simulator Documentation}
\subsubsection{Events and State}
\subsubsection{How to change things}

 
\end{document}